\documentclass[12pt]{article}
\usepackage[utf8]{inputenc}
\usepackage{fancyhdr}
\usepackage{geometry}
\usepackage{newunicodechar}
\newunicodechar{₂}{$_2$}
\geometry{margin=1in}

% Header and Footer setup
\pagestyle{fancy}
\fancyhf{}
\fancyhead[C]{The Rising Heat: A Global Warning We Can’t Ignore}
\fancyfoot[C]{Page \thepage}

\title{\textbf{An Overview of Indian Economy: Growth, Challanges and Oppurtunity} \\[1ex]
\large Maulana Abul Kalam Azad University}


\begin{document}

\maketitle

\section*{Section 1: Overview of the Indian Economy}

The Indian economy is one of the largest and fastest-growing economies in the world. It operates as a mixed economy, combining features of both capitalism and socialism. Agriculture, industry, and services form the three main sectors. Over the past few decades, India has seen rapid growth in its service sector, especially in information technology and telecommunications. Economic reforms introduced in 1991 played a key role in liberalizing the economy, encouraging foreign investment, and enhancing global trade relationships. Despite challenges like poverty and unemployment, India continues to move towards becoming a $5 trillion economy.

\section*{Section 2: Current Challenges and Opportunities}

India's economic growth is accompanied by several challenges such as income inequality, inflation, rural distress, and unemployment. Additionally, external factors like global inflation and oil prices impact the nation's financial stability. However, there are immense opportunities for development in areas like renewable energy, digital transformation, infrastructure, and startups. Government initiatives like Make in India, Digital India, and Startup India aim to foster innovation, boost manufacturing, and create employment. With a young population, a growing digital economy, and policy reforms, India is poised for a resilient and inclusive economic future.

\end{document}
\end{document}